% Metódy inžinierskej práce

\documentclass[10pt,british,a4paper,titlepage]{article}

\usepackage[english]{babel}
%\usepackage[T1]{fontenc}
\usepackage[IL2]{fontenc} 
\usepackage[utf8]{inputenc}
\usepackage{graphicx}
\usepackage{url} 
\usepackage{hyperref} 

\usepackage{cite}
%\usepackage{times}

\pagestyle{headings}

\title{Revenue and monetization of the video game industry\thanks{Semestrálny projekt v predmete Metódy inžinierskej práce, ak. rok 2022/23 \\vedenie: Mirwais Ahmadzai}}

\author{Tomáš Tuan Bui Anh\\[2pt]
	{\small Slovenská technická univerzita v Bratislave}\\
	{\small Fakulta informatiky a informačných technológií}\\
	{\small \texttt{xbuianh@stuba.sk}}
	}

\date{\small 23. október 2022}


\begin{document}

\maketitle

\begin{abstract}
This article will investigate, evaluate, and compare the changing trends of the video game industry or VGI for short, its evolution from a relatively minor market intended towards a specific demographic to an economic giant, comparable in revenue to other branches of the entertainment industry. Main talking points include:
\begin{enumerate}
\item VGI revenue and transition from direct game sales to micro-transactions and in game markets
\item Expansion to the mobile market
\item Results of the extended access of games to society 
\end{enumerate}
\end{abstract}



\section{Introduction}

Games have always fascinated and influenced both individuals and society. They provide ample relaxation in our lives, of which we can afford so little, making them important for us to stay content hence the saying “Bread and Circuses”. It comes as no surprise that with the rising standard of living and the increased amount of free time, we are spending more and more time engaging in games, one way or another. One such branch is the video game industry, a relatively new subgroup of entertainment and completely revolutionary in terms of scale, availability and replayability when compared to traditional games. Since the invention, then further improvement and increased availability of computers ,especially laptops, and game consoles, it has witnessed a rapid growth from a relatively minor market intended towards a specific demographic to an economic giant, comparable in revenue to the likes of film industry. However, with growth also came change, particularly in business model. That, together with video game industry revenue and the relatively modern micro-transaction model, will be the main talking of this article.



\section{Business Models} 
Business model, referred to as BM henceforth, generally refers to a plan or a strategy aimed at maximizing revenue, identifying the key customer base, products and means of financing. The rapid expansion of the industry resulted in the creation of a highly competitive environment, where companies use their limited resources to out-compete their rivals\cite{osathanunkul2015classification:business}. Therefore, different business models became crucial to thrive, especially for smaller companies. Increased development costs and time constraints have forced firms to look for alternative strategies to run the business and increase revenue. New technology and innovations such as subscriptions and micro-transactions conditioned these radical changes in business models of the VDI\cite{osathanunkul2015classification:business}.



\subsection{Classification of BM by accessibility of the customer}
\begin{enumerate} 

\item Pay-to-Play (P2P)
Considered to be one of the first BM, it is a traditional model in VDI, in which the customer must make a payment before they can access most of the product or service. In VDI, Massively Multiplayer Online Role-Playing Games or MMORPGs often require continuous payment from players to maintain a playing account, as is the case in World of Warcraft, arguably the most popular MMORPG of all time \cite{IGN:WOW}.

\item Free-to-Play (F2P) 
Unlike P2P, the F2P model allows the user substantial access to service and/or content free of charge before they must pay for any additional features. Since this BM concept was popularized, many sub models were created, such as Shareware, Freeware or Open source.

Shareware is a trial version given free of charge, usually limited in time, and/or functionality until the customer pays for the full version. In VDI, this model is generally called a video game demo\cite{osathanunkul2015classification:business}.

Freeware is software which is available without any necessary payment or only with an optional payment. Typically, Freeware is provided without source code, author retains all other rights, including the rights to copy, distribute, and make derivative works from the software\cite{Graham:LBCI}.

Open source is a freeware that provides users with the source of the software, allowing them to substantially modify it. 

\end{enumerate}




\section{Brief history of VDI and primary sources of revenue.}

Accessibility has a direct impact on the primary source of revenue, which changed radically throughout history.
 
\begin{itemize}

\item One of the earliest forms of electronic games were arcade video game machines. Characterized by their simplistic nature, rules and near infinite replayability, they were highly addictive\cite{gao2022nature:arcade}. The monetization method of these machines is coin-operated, where users must insert a coin to activate the machine. However, with the improvements and increased affordability of video game consoles and computers, their use declined \cite{osathanunkul2015classification:business}.

\item Once consoles and computers became widespread in society, games were distributed in physical form usually through CDs, cartridges, cassettes or other types of discs. These could be purchased in mainly dedicated video game stores  or other electronics stores. The revenue came mainly from direct sales (P2P BM).

\item To maximize the revenue companies began experimenting with digital distribution which was made more available through developments in network bandwidth capabilities, mainly the download speed. As such, it quickly gained a significant share in the VDI market \cite{osathanunkul2015classification:business}. At first, the digital distribution was practiced in parallel with the physical one, but after further developments in technology, the use of physical distribution steadily declined and was made obsolete, which is proven by the lack of CD-ROMs on most computers.

\item To further increase revenue, some companies began implementing subscription systems, where users must periodically pay to have continued access and to be able to play the game. This allowed companies to further charge the users after the initial payment was made. 

\end{itemize}
Further development of this model resulted in the creation of micro-transaction model commonly used today, which will be further explored in the next section.



\section{Microtransactions}

Microtransactions generally refer to paying for some form of additional in-game content. They usually tend to be low-amount payments, amounting to only a few euros. To attract as many people as possible, companies oftentimes offer games for only a trivial price or free altogether. Then, additional content is added to the base game, which either enhances it, makes some changes in its design, removing in-game limitations or simply serves a cosmetical purpose. This additional content can be purchased at a relatively low price. This concept, which came from the mobile game market that is itself a branch of VDI, quickly settled in its predecessors’ core models and redefined the entire main VDI business model. The concept was originally used by small independent developer teams known as indie developers, to boost their own revenue and to be more competitive against mainstream publishers with larger budgets. This trend quickly caught on with the publishers, who in turn changed their own business policies and began implementing and relying on microtransactions themselves, even in big mainstream games where users paid full price when buying. Microtransactions differ from game to game, depending mainly on its genre. Users can buy various additional content ranging from in-game cosmetics to additional content like DLCs. 

In-game cosmetics, themselves range from simple added effects to complete reworks in artistic design such as alternative costume, added voice lines for in-game character or additional animations. These effects generally have no effect on gameplay, which results in mixed responses from the gaming community. On one hand, as in-game cosmetics have no effect on gameplay, especially in multiplayer games, they are accepted more than other microtransactions that do have an effect. On the other hand, parts of the gaming community have started seeing them as a clever ploy to keep the players interested in the game and  a tool to further exploit the gaming community and to boost revenue. These microtransactions can nowadays be found in most games, regardless of whether they are multiplayer or singleplayer. 

Downloadable content or DLC for short, adds substantial amount of content to the existing game. These expansion packs, which contain new storylines, new characters, new gameplay mechanics or other new content that affects gameplay, are often sold at a price like that of the original game. 

As most players won’t buy microtransactions, publishers attempt to attract a large enough audience of those who are willing to invest large amounts of money, therefore increasing their income. A 2014 study found that only 1.5\% of F2P games players used microtransactions and 10\% of them produce more than 50\% of all the revenue from microtransactions\cite{tomic2018economic:micro}.



\subsection{Implementation of microtransactions in VDI}

Microtransactions are implemented through either direct purchase using real currency or through an intermediary in-game currency purchased with real money. This has a psychological effect on users as it makes in-game prices more abstract, therefore making it more difficult to discern the real price of the product or its relative price compared to other products\cite{tomic2018economic:micro}.

Most games with microtransactions utilize an in-game currency in the form of generic currencies used extensively like gold coins , gems, or currencies unique for a particular game title or franchise. A single game can have multiple in-game currencies, some of which can be earned through playing, whilst others can only be bought with real money. Games with multiple currencies often have premium currencies, purchasable only with real money, which come in different shapes and forms. This in-game currency is then spent on additional content.



\subsection{Lootboxes}

The other frequent microtransactions model are so-called lootboxes. These can be bought, for either real or in-game currency, gained through playing, or simply received on special occasions. Lootboxes also differ in accessibility, in some games, a lootbox can be redeemed immediately, whereas in others, the user must “open” them with a “key”. The content of lootboxes varies from cosmetics, which have no gameplay effect, to in-game boost and modifications that can make gameplay easier. 

Even though all forms of microtransactions are generally looked down upon, lootboxes have been the subject of legal consideration in several countries (Microtransactions). Opening lootboxes is essentially gambling, as the user doesn’t know beforehand its content. Therefore, although the cost of opening a single lootbox is generally low, its content might not be what the user wanted, prompting them to open more. For example, FIFA 18 card packs, a variation of lootbox, was deemed by the Belgian court to represent a form of gambling. This resulted in ban on selling lootboxes in Belgium for several publishers\cite{tomic2018economic:micro}.




\section{Mobile gaming} 


Although, hand-held consoles existed prior to popularization of smart phones, their use was usually limited to gaming, whilst older cellphones were used primarily for work and other non-gaming related activities. However, with the introduction of technology like smart phones and their further developments, these two activities could be carried out on a single device. 

According to UN's International Telecommunications Union, at the end of 2018, mobile devices outnumbered the population of Earth. This allowed VDI companies to penetrate this new market of potential users\cite{mayra2020mobile:mob}.   

One of the main differences between mobile gaming and computer or console gaming is in monetization and the BM used. Most mobile games are F2P, relying on in-game microtransactions and as such are developed accordingly.

They are typically never ending, having infinite replayability, and being updated on regular basis so as to keep the game interesting. Players are encouraged to pay for additional content to advance faster, or to remove certain limiting mechanics like advertisement between gameplay. For example, in \emph{Clash of Clans} players can use premium currency known as gems to speed up progress of training new troops. To prevent players from advancing too quickly and therefore loosing interest in the game, time restrictions are implemented such as energy which depletes by playing or waiting for an action to finish\cite{mayra2020mobile:mob}. 

\section{Effects of video games on individuals}

Apart from causing addiction, video games can also be an expensive hobby. As we discovered,in both F2P and P2P, apart from initial direct sales, companies focus their efforts on cultivating a small percentage of high-paying player who account for more 50\% percent of microtransactions revenue. The alternative of paying to progress is "grinding" the game, which is time-consuming. This results in lost time for dedicated F2P players.   


\section{Conclusion}

This article classifies business models used in VDI, compares their main sources of revenue and briefly examines their history and history of video games themselves. 

Furthermore, the article examines microtransactions. The negative attitude and backlash from gaming community and authorities, classification of microtransactions, their correlation with gambling and their impact on our finances and free time.  

Lastly, mobile games are discussed, their breakthrough and remarkable growth, and their comparison with computer and console games.   




\bibliography{references}
\bibliographystyle{plain}


\end{document}
