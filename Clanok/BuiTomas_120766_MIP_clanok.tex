% Metódy inžinierskej práce

\documentclass[10pt,british,a4paper,titlepage]{article}

\usepackage[english]{babel}
%\usepackage[T1]{fontenc}
\usepackage[IL2]{fontenc} % lepšia sadzba písmena Ľ než v T1
\usepackage[utf8]{inputenc}
\usepackage{graphicx}
\usepackage{url} % príkaz \url na formátovanie URL
\usepackage{hyperref} % odkazy v texte budú aktívne (pri niektorých triedach dokumentov spôsobuje posun textu)

\usepackage{cite}
%\usepackage{times}

\pagestyle{headings}

\title{Revenue and monetization of the video game industry and its effects on society\thanks{Semestrálny projekt v predmete Metódy inžinierskej práce, ak. rok 2022/23 \\vedenie: Mirwais Ahmadzai}} % meno a priezvisko vyučujúceho na cvičeniach

\author{Tomáš Tuan Bui Anh\\[2pt]
	{\small Slovenská technická univerzita v Bratislave}\\
	{\small Fakulta informatiky a informačných technológií}\\
	{\small \texttt{xbuianh@stuba.sk}}
	}

\date{\small 23. október 2022} % upravte



\begin{document}

\maketitle

\begin{abstract}
This article will investigate, evaluate, and compare the changing trends of the video game industry or VGI for short, its evolution from a relatively minor market intended towards a specific demographic to an economic giant, comparable in revenue to other branches of the entertainment industry. Main talking points include:
\begin{enumerate}
\item VGI revenue and transition from direct game sales to micro-transactions and in game markets
\item Expansion to the mobile market
\item Results of the extended access of games to society 
\end{enumerate}
\end{abstract}



\section{Introduction}

Games have always fascinated and influenced both individuals and society. They provide ample relaxation in our lives, of which we can afford so little, making them important for us to stay content hence the saying “Bread and Circuses”. It comes as no surprise that with the rising standard of living and the increased amount of free time, we are spending more and more time engaging in games, one way or another. One such branch is the video game industry, a relatively new subgroup of entertainment and completely revolutionary in terms of scale, availability and replayability when compared to traditional games. Since the invention, then further improvement and increased availability of computers ,especially laptops, and game consoles, it has witnessed a rapid growth from a relatively minor market intended towards a specific demographic to an economic giant, comparable in revenue to the likes of film industry. However, with growth also came change, particularly in business model. That, together with video game industry revenue and the relatively modern micro-transaction model, will be the main talking of this article.



\section{Business Models} 
Business model, referred to as BM henceforth, generally refers to a plan or a strategy aimed at maximizing revenue, identifying the key customer base, products and means of financing. The rapid expansion of the industry resulted in the creation of a highly competitive environment, where companies use their limited resources to out-compete their rivals\cite{osathanunkul2015classification:business}. Therefore, different business models became crucial to thrive, especially for smaller companies. Increased development costs and time constraints have forced firms to look for alternative strategies to run the business and increase revenue. New technology and innovations such as subscriptions and micro-transactions conditioned these radical changes in business models of the VDG\cite{osathanunkul2015classification:business}.



\subsection{Classification of BM by accessibility of the customer}
\begin{enumerate} 

\item Pay-to-Play (P2P)
Considered to be one of the first BM, it is a traditional model in VDG, in which the customer must make a payment before they can access most of the product or service. In VDG, Massively Multiplayer Online Role-Playing Games or MMORPGs often require continuous payment from players to maintain a playing account, as is the case in World of Warcraft, arguably the most popular MMORPG of all time (IGN).

\item Free-to-Play (F2P) 
Unlike P2P, the F2P model allows the user substantial access to service and/or content free of charge before they must pay for any additional features. Since this BM concept was popularized, many sub models were created, such as Shareware, Freeware or Open source.

Shareware is a trial version given free of charge, usually limited in time, and/or functionality until the customer pays for the full version. In VDG, this model is generally called a video game demo\cite{osathanunkul2015classification:business}.

Freeware is software which is available without any necessary payment or only with an optional payment. Typically, Freeware is provided without source code, author retains all other rights, including the rights to copy, distribute, and make derivative works from the software\cite{Graham:LBCI}.

Open source is a freeware that provides users with the source of the software, allowing them to substantially modify it. 

\end{enumerate}




\section{Brief history of VDG and primary sources of revenue.}

Accessibility has a direct impact on the primary source of revenue, which changed radically throughout history.
 
\begin{itemize}

\item One of the earliest forms of electronic games were arcade video game machines. Characterized by their simplistic nature, rules and near infinite replayability, they were highly addictive (Nature of arcade games). The monetization method of these machines is coin-operated, where users must insert a coin to activate the machine. However, with the improvements and increased affordability of video game consoles and computers, their use declined (circle international).

\item Once consoles and computers became widespread in society, games were distributed in physical form usually through CDs, cartridges, cassettes or other types of discs. These could be purchased in mainly dedicated video game stores  or other electronics stores. The revenue came mainly from direct sales (P2P BM).

\item To maximize the revenue companies began experimenting with digital distribution which was made more available through developments in network bandwidth capabilities, mainly the download speed. As such, it quickly gained a significant share in the VDG market (circle international). At first, the digital distribution was practiced in parallel with the physical one, but after further developments in technology, the use of physical distribution steadily declined and was made obsolete, which is proven by the lack of CD-ROMs on most computers.

\item To further increase revenue, some companies began implementing subscription systems, where users must periodically pay to have continued access and to be able to play the game. This allowed companies to further charge the users after the initial payment was made. 

\end{itemize}
Further development of this model resulted in the creation of micro-transaction model commonly used today, which will be further explored in the next section.

\subsection{Nejaké vysvetlenie} \label{ina:nejake}

Niekedy treba uviesť zoznam:

\begin{itemize}
\item jedna vec
\item druhá vec
	\begin{itemize}
	\item x
	\item y
	\end{itemize}
\end{itemize}

Ten istý zoznam, len číslovaný:

\begin{enumerate}
\item jedna vec
\item druhá vec
	\begin{enumerate}
	\item x
	\item y
	\end{enumerate}
\end{enumerate}


\subsection{Ešte nejaké vysvetlenie} \label{ina:este}

\paragraph{Veľmi dôležitá poznámka.}
Niekedy je potrebné nadpisom označiť odsek. Text pokračuje hneď za nadpisom.



\section{Dôležitá časť} \label{dolezita}




\section{Ešte dôležitejšia časť} \label{dolezitejsia}




\section{Záver} \label{zaver} % prípadne iný variant názvu



%\acknowledgement{Ak niekomu chcete poďakovať\ldots}


% týmto sa generuje zoznam literatúry z obsahu súboru literatura.bib podľa toho, na čo sa v článku odkazujete
\bibliography{references}
\bibliographystyle{plain} % prípadne alpha, abbrv alebo hociktorý iný


\end{document}
